\documentclass[11pt]{amsart}
\usepackage{amssymb,amsmath,amsthm,mathrsfs}
\usepackage[margin=.75in]{geometry}
\usepackage{enumitem}
\usepackage{tikz}
\usetikzlibrary{arrows,calc}
\tikzset{>=stealth}


%%%%%%%%%%%%%%%%%%%%%%%%%%%%%%%%%%%%%%%%%%%%%%%
%%%%%%%%%%%%%%%% NEW COMMANDS %%%%%%%%%%%%%%%%%

%Sets
\newcommand{\A}{\mathbb{A}}
\newcommand{\B}{\mathbb{B}}
\newcommand{\C}{\mathbb{C}}
\newcommand{\F}{\mathbb{F}}
\newcommand{\K}{\mathbb{K}}
\newcommand{\M}{\mathbb{M}}
\newcommand{\N}{\mathbb{N}}
\renewcommand{\P}{\mathbb{P}}
\newcommand{\Q}{\mathbb{Q}}
\newcommand{\R}{\mathbb{R}}
\newcommand{\Z}{\mathbb{Z}}
\newcommand{\card}{\operatorname{card}}

%Group Theory
\newcommand{\GL}{\operatorname{GL}}
\newcommand{\PU}{\operatorname{PU}}
\newcommand{\SL}{\operatorname{SL}}
\newcommand{\SU}{\operatorname{SU}}
\newcommand{\U}{\operatorname{U}}

%Ring Theory
\newcommand{\Char}{\operatorname{Char}}

%Field Theory
\newcommand{\gal}{\operatorname{Gal}}
\newcommand\aut[2]{\operatorname{Aut}_{{}_{#1}} ({#2})}
\newcommand\irr[3]{\operatorname{Irr}_{{}_{#1}}({#2},{#3})}

%Module Theory
\newcommand{\Ann}{\operatorname{Ann}}
\newcommand{\Ch}{\operatorname{Ch}}
\newcommand{\coim}{\operatorname{coim}}
\newcommand{\coker}{\operatorname{coker}}
\newcommand{\Hom}{\operatorname{Hom}}

%Misc
\renewcommand\bar[1]{\overline{#1}}
\newcommand\Id{\operatorname{Id}}
\newcommand\im{\operatorname{im}}
\newcommand\lcm{\operatorname{lcm}}
\newcommand\nr[2]{\operatorname{N}_{{}_{#1}} ({#2})}
\newcommand\tr[2]{\operatorname{Tr}_{{}_{#1}} ({#2})}

%Font Style
\newcommand\ds[1]{{\displaystyle #1}}
\newcommand\mc[1]{\mathcal{#1}}
\newcommand\mf[1]{\mathfrak{#1}}
\newcommand\ms[1]{\mathscr{#1}}
\newcommand\ssty[1]{{\scriptstyle #1}}
\newcommand\sssty[1]{{\scriptscriptstyle #1}}

%Theorem Stuffs
\theoremstyle{plain}
\newtheorem{thm}{Theorem}
\newtheorem{lemma}{Lemma}
\newtheorem{prob}{Problem}
\newtheorem{defn}{Definition}
\newtheorem{prop}{Proposition}
\newtheorem{cor}{Corollary}

\theoremstyle{definition}
\newtheorem{conj}{Conjecture}
\newtheorem*{ex}{Example}
\newtheorem{alg}{Algorithm}
\newtheorem{exc}{Problem}

\theoremstyle{remark}
\newtheorem*{remark}{Remark}
\newtheorem*{note}{Note}
\newtheorem{case}{Case}

%%%%%%%%%%%%%%%%%%%%%%%%%%%%%%%%%%%%%%%%%%%%%%%
%%%%%%%%%%%%%%%%%%%%%%%%%%%%%%%%%%%%%%%%%%%%%%%




\title{Is this object a circle?}
\author{}
\date{}

\begin{document}
\maketitle



Given a simple closed curve in the plane $S$, we seek simple
algorithms that will identify $S$ as not being a circle. Specifically,
we wish to identify noncircles via algorithms that measure chords.

We'll identify a simple closed curve $C\in\R^2$ as a circle if
\[
C\subset \{x^2+y^2 = 1 + \epsilon \}
\]
where $\epsilon$ is a numerical tolerance on the circle. Given
\[
S\not\subset \{x^2+y^2 = 1 + \epsilon \}
\]
we wish to examine the methods of determining this non-inclusion by
\begin{enumerate}
\item Measuring relative arc length.
\item Measuring angles.
\item Meauring chords.
\item Measuring widths.
\end{enumerate}

Suppose we fish some device out of the ocean with a face whose
perimeter is given by a simple closed curve, \(S\).  To what degree
can we determine if \(S\) is a circle by measuring chords and
diameters?


\begin{thebibliography}{9}
\bibitem[label1]{aR2003} A.\ Ricotta, ``Constant-Diameter Curves,'' The Mathematical
Intelligencer \textbf{25}, no. 4, 2003, 4--5.
\end{thebibliography}
\end{document}








